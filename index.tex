% Options for packages loaded elsewhere
\PassOptionsToPackage{unicode}{hyperref}
\PassOptionsToPackage{hyphens}{url}
\PassOptionsToPackage{dvipsnames,svgnames,x11names}{xcolor}
%
\documentclass[
  letterpaper,
  DIV=11,
  numbers=noendperiod]{scrreprt}

\usepackage{amsmath,amssymb}
\usepackage{iftex}
\ifPDFTeX
  \usepackage[T1]{fontenc}
  \usepackage[utf8]{inputenc}
  \usepackage{textcomp} % provide euro and other symbols
\else % if luatex or xetex
  \usepackage{unicode-math}
  \defaultfontfeatures{Scale=MatchLowercase}
  \defaultfontfeatures[\rmfamily]{Ligatures=TeX,Scale=1}
\fi
\usepackage{lmodern}
\ifPDFTeX\else  
    % xetex/luatex font selection
\fi
% Use upquote if available, for straight quotes in verbatim environments
\IfFileExists{upquote.sty}{\usepackage{upquote}}{}
\IfFileExists{microtype.sty}{% use microtype if available
  \usepackage[]{microtype}
  \UseMicrotypeSet[protrusion]{basicmath} % disable protrusion for tt fonts
}{}
\makeatletter
\@ifundefined{KOMAClassName}{% if non-KOMA class
  \IfFileExists{parskip.sty}{%
    \usepackage{parskip}
  }{% else
    \setlength{\parindent}{0pt}
    \setlength{\parskip}{6pt plus 2pt minus 1pt}}
}{% if KOMA class
  \KOMAoptions{parskip=half}}
\makeatother
\usepackage{xcolor}
\setlength{\emergencystretch}{3em} % prevent overfull lines
\setcounter{secnumdepth}{5}
% Make \paragraph and \subparagraph free-standing
\makeatletter
\ifx\paragraph\undefined\else
  \let\oldparagraph\paragraph
  \renewcommand{\paragraph}{
    \@ifstar
      \xxxParagraphStar
      \xxxParagraphNoStar
  }
  \newcommand{\xxxParagraphStar}[1]{\oldparagraph*{#1}\mbox{}}
  \newcommand{\xxxParagraphNoStar}[1]{\oldparagraph{#1}\mbox{}}
\fi
\ifx\subparagraph\undefined\else
  \let\oldsubparagraph\subparagraph
  \renewcommand{\subparagraph}{
    \@ifstar
      \xxxSubParagraphStar
      \xxxSubParagraphNoStar
  }
  \newcommand{\xxxSubParagraphStar}[1]{\oldsubparagraph*{#1}\mbox{}}
  \newcommand{\xxxSubParagraphNoStar}[1]{\oldsubparagraph{#1}\mbox{}}
\fi
\makeatother


\providecommand{\tightlist}{%
  \setlength{\itemsep}{0pt}\setlength{\parskip}{0pt}}\usepackage{longtable,booktabs,array}
\usepackage{calc} % for calculating minipage widths
% Correct order of tables after \paragraph or \subparagraph
\usepackage{etoolbox}
\makeatletter
\patchcmd\longtable{\par}{\if@noskipsec\mbox{}\fi\par}{}{}
\makeatother
% Allow footnotes in longtable head/foot
\IfFileExists{footnotehyper.sty}{\usepackage{footnotehyper}}{\usepackage{footnote}}
\makesavenoteenv{longtable}
\usepackage{graphicx}
\makeatletter
\def\maxwidth{\ifdim\Gin@nat@width>\linewidth\linewidth\else\Gin@nat@width\fi}
\def\maxheight{\ifdim\Gin@nat@height>\textheight\textheight\else\Gin@nat@height\fi}
\makeatother
% Scale images if necessary, so that they will not overflow the page
% margins by default, and it is still possible to overwrite the defaults
% using explicit options in \includegraphics[width, height, ...]{}
\setkeys{Gin}{width=\maxwidth,height=\maxheight,keepaspectratio}
% Set default figure placement to htbp
\makeatletter
\def\fps@figure{htbp}
\makeatother
% definitions for citeproc citations
\NewDocumentCommand\citeproctext{}{}
\NewDocumentCommand\citeproc{mm}{%
  \begingroup\def\citeproctext{#2}\cite{#1}\endgroup}
\makeatletter
 % allow citations to break across lines
 \let\@cite@ofmt\@firstofone
 % avoid brackets around text for \cite:
 \def\@biblabel#1{}
 \def\@cite#1#2{{#1\if@tempswa , #2\fi}}
\makeatother
\newlength{\cslhangindent}
\setlength{\cslhangindent}{1.5em}
\newlength{\csllabelwidth}
\setlength{\csllabelwidth}{3em}
\newenvironment{CSLReferences}[2] % #1 hanging-indent, #2 entry-spacing
 {\begin{list}{}{%
  \setlength{\itemindent}{0pt}
  \setlength{\leftmargin}{0pt}
  \setlength{\parsep}{0pt}
  % turn on hanging indent if param 1 is 1
  \ifodd #1
   \setlength{\leftmargin}{\cslhangindent}
   \setlength{\itemindent}{-1\cslhangindent}
  \fi
  % set entry spacing
  \setlength{\itemsep}{#2\baselineskip}}}
 {\end{list}}
\usepackage{calc}
\newcommand{\CSLBlock}[1]{\hfill\break\parbox[t]{\linewidth}{\strut\ignorespaces#1\strut}}
\newcommand{\CSLLeftMargin}[1]{\parbox[t]{\csllabelwidth}{\strut#1\strut}}
\newcommand{\CSLRightInline}[1]{\parbox[t]{\linewidth - \csllabelwidth}{\strut#1\strut}}
\newcommand{\CSLIndent}[1]{\hspace{\cslhangindent}#1}

\KOMAoption{captions}{tableheading}
\makeatletter
\@ifpackageloaded{bookmark}{}{\usepackage{bookmark}}
\makeatother
\makeatletter
\@ifpackageloaded{caption}{}{\usepackage{caption}}
\AtBeginDocument{%
\ifdefined\contentsname
  \renewcommand*\contentsname{Table of contents}
\else
  \newcommand\contentsname{Table of contents}
\fi
\ifdefined\listfigurename
  \renewcommand*\listfigurename{List of Figures}
\else
  \newcommand\listfigurename{List of Figures}
\fi
\ifdefined\listtablename
  \renewcommand*\listtablename{List of Tables}
\else
  \newcommand\listtablename{List of Tables}
\fi
\ifdefined\figurename
  \renewcommand*\figurename{Figure}
\else
  \newcommand\figurename{Figure}
\fi
\ifdefined\tablename
  \renewcommand*\tablename{Table}
\else
  \newcommand\tablename{Table}
\fi
}
\@ifpackageloaded{float}{}{\usepackage{float}}
\floatstyle{ruled}
\@ifundefined{c@chapter}{\newfloat{codelisting}{h}{lop}}{\newfloat{codelisting}{h}{lop}[chapter]}
\floatname{codelisting}{Listing}
\newcommand*\listoflistings{\listof{codelisting}{List of Listings}}
\makeatother
\makeatletter
\makeatother
\makeatletter
\@ifpackageloaded{caption}{}{\usepackage{caption}}
\@ifpackageloaded{subcaption}{}{\usepackage{subcaption}}
\makeatother

\ifLuaTeX
  \usepackage{selnolig}  % disable illegal ligatures
\fi
\usepackage{bookmark}

\IfFileExists{xurl.sty}{\usepackage{xurl}}{} % add URL line breaks if available
\urlstyle{same} % disable monospaced font for URLs
\hypersetup{
  pdftitle={SoTL Compendium},
  pdfauthor={Helena Paterson; Phil McAleer; James Bartlett},
  colorlinks=true,
  linkcolor={blue},
  filecolor={Maroon},
  citecolor={Blue},
  urlcolor={Blue},
  pdfcreator={LaTeX via pandoc}}


\title{SoTL Compendium}
\author{Helena Paterson \and Phil McAleer \and James Bartlett}
\date{2025-04-02}

\begin{document}
\maketitle

\renewcommand*\contentsname{Table of contents}
{
\hypersetup{linkcolor=}
\setcounter{tocdepth}{2}
\tableofcontents
}

\bookmarksetup{startatroot}

\chapter*{Overview}\label{overview}
\addcontentsline{toc}{chapter}{Overview}

\markboth{Overview}{Overview}

The glossary outlines a crowd-sourced collection of terms used in the
Scholarship of Teaching and Learning (SoTL).

\bookmarksetup{startatroot}

\chapter{A}\label{a}

\section{Accreditation}\label{accreditation}

A quality assurance and enhancement mechanism for programmes/courses by
a relevant institution/external professional body, to ensure standards.

Long Definition:

Deep Dive:

Accreditation can mean different things/be applied differently in
different academic disciplines.
https://www.qaa.ac.uk/reviewing-higher-education/degree-awarding-powers-and-university-title
Some schools/departments within a university will be accredited by an
organisation (for example the Adam Smith Business School at University
of Glasgow is accredited by AACSB, EQUIS, and AMBA) whereas in other
areas some degrees are accredited by professional bodies/organisations.
This means that they are aligned with the professional body's own
qualifications and ensures that the education programme meets industry
standards. Examples include the City Planning and Real Estate Msc
degrees at the University of Glasgow are accredited by the
\href{http://www.rtpi.org.uk\%20https://www.rtpi.org.uk/become-a-planner/study-at-university/wales-scotland-and-northern-ireland/}{Royal
Town Planning Institute (RTPI)} and the
\href{https://www.rics.org\%20https://www.rics.org/surveyor-careers/how-to-become-a-surveyor/university-surveying-courses}{Royal
Institute for Chartered Surveyors} Other example include medicine and
dentistry.

Contributors: Joanna Stewart, Alison McCandlish

Tags: NA

\section{Active Learning}\label{active-learning}

Learning by doing and reflection; this could be individual or
collaborative.

Long Definition:

Active learning is defined as ``any instructional method that engages
students in the learning process\ldots.{[}it{]} requires students to do
meaningful activities and think about what they are doing'' (prince,
2004, p.233, citing bonwell and eison 1991).

Deep Dive:

Active learning can take place in any format including online or
in-person. There is a large variety of methods and strategies that can
be employed. e.g., team-based learning, think, pair, share and
problem-based learning. Active learning supports various skills such as
critical thinking, problem-solving, team-work, communication and
collaboration. Active learning facilitates deep learning, vs superficial
learning.

Contributors: Vicki Dale

Tags: NA

\section{Ai Driven Adaptive Learning}\label{ai-driven-adaptive-learning}

-- Ai driven systems provide a more personalised learning experience
which cater to diverse learning preferences/requirements.

Long Definition:

Ai driven adaptive learning enhances student engagement and performance,
scaffolding individual learner pathways in real-time based on
performance and preference. It requires technology to diagnose a
starting point for learners in terms of their existing knowledge and
skills. Subsequently upskilling them and continually monitoring their
performance. The systems provide immediate feedback to the students
which informs their subsequent learning tasks.

Deep Dive:

NA

Contributors: Vicki Dale

Tags: NA

\section{Assessment}\label{assessment}

Method for evaluating learning and skills of learners

Long Definition:

Various methods for evaluating the learning that has taken place during
a course, module or programme of study. Typically centered on
demonstrating how the intended learning objectives of a course has been
met, but also includes skills in communication that a student developed
through a course

Deep Dive:

While the goal is often to measure the attainment of learning
objectives, something teachers forget to do is to align their assessment
with those objectives or in other cases the method of assessment can
rely on a student needing to develop skills in communication that are
assumed/not explicitly taught. The process of constructive alignment can
help to align

Contributors: Helena Paterson

Tags: NA

\section{Asynchronous}\label{asynchronous}

Occurring at different times. Asynchronous learning refers to students
accessing materials at their own pace/in their own time (for example
watching pre-recorded lectures, or doing readings).

Long Definition:

Deep Dive:

NA

Contributors: Joanna Stewart

Tags: NA

\bookmarksetup{startatroot}

\chapter{C}\label{c}

\section{Co-Creation}\label{co-creation}

NA

Long Definition:

Deep Dive:

NA

Contributors: NA

Tags: NA

\section{Coil}\label{coil}

An abbreviation for collaborative online international learning

Long Definition:

Deep Dive:

NA

Contributors: NA

Tags: NA

\section{Consent}\label{consent}

Voluntary agreement to take part in an event

Long Definition:

Deep Dive:

NA

Contributors: NA

Tags: NA

\section{Control Groups}\label{control-groups}

NA

Long Definition:

Deep Dive:

NA

Contributors: NA

Tags: NA

\bookmarksetup{startatroot}

\chapter{D}\label{d}

\section{Demographics}\label{demographics}

The characteristics of a population. (often described with descriptive
statistics). With regard to students key demographics include gender,
age, socio-economic background, country of origin, ethnicity, employment
status, caring responsibilities.

Long Definition:

Deep Dive:

NA

Contributors: Joanna Stewart

Tags: NA

\section{Digital Credentials}\label{digital-credentials}

NA

Long Definition:

Deep Dive:

NA

Contributors: NA

Tags: NA

\section{Dissemination}\label{dissemination}

NA

Long Definition:

Deep Dive:

NA

Contributors: NA

Tags: NA

\bookmarksetup{startatroot}

\chapter{E}\label{e}

\section{Ecopedagogy}\label{ecopedagogy}

NA

Long Definition:

Deep Dive:

NA

Contributors: NA

Tags: NA

\section{Edi}\label{edi}

An abbreviation for the term equality, diversity and inclusion

Long Definition:

Deep Dive:

NA

Contributors: NA

Tags: NA

\section{Effect Sizes}\label{effect-sizes}

A measure of the outcome of a study.

Long Definition:

Deep Dive:

NA

Contributors: NA

Tags: NA

\section{Epistemology}\label{epistemology}

Epistemology is the study of the nature of knowledge, how it is defined,
what can be known, and what are its limits.

Long Definition:

Epistemology is the study of the nature of knowledge, how it is defined,
what can be known, and what are its limits.

Guba, e. S. and lincoln, y. S. (1994) `competing paradigms in
qualitative research', in denzin, n.k. \& Lincoln, y.s. (eds.) Handbook
of qualitative research. 2 ed.~Thousand oaks, sage.

``Epistemology, according to the oxford english dictionary, is the
theory or science of the method and ground of knowledge. It is a core
area of philosophical study that includes the sources and limits,
rationality and justification of knowledge.''

Stone, l. 2008. Epistemology. In given, l. (ed.) The sage encyclopedia
of qualitative research methods. Thousand oaks.

``Branch of philosophy that investigates the possibility, origins,
nature, and extent of human knowledge. Although the effort to develop an
adequate theory of knowledge is at least as old as plato's theaetetus,
epistemology has dominated western philosophy only since the era of
descartes and locke, as an extended dispute between rationalism and
empiricism over the respective importance of a priori and a posteriori
origins. Contemporary postmodern thinkers (including many feminist
philosophers) have proposed the contextualization of knowledge as part
of an intersubjective process.

``Defined narrowly, epistemology is the study of knowledge and justified
belief. As the study of knowledge, epistemology is concerned with the
following questions. What are the necessary and sufficient conditions of
knowledge? What are its sources? What is its structure, and what are its
limits? As the study of justified belief, epistemology aims to answer
questions such as how we are to understand the concept of justification?
What makes justified beliefs justified? Is justification internal or
external to one's own mind? Understood more broadly, epistemology is
about issues having to do with the creation and dissemination of
knowledge in particular areas of inquiry.

Deep Dive:

NA

Contributors: Nic Kipar

Tags: Methods

\section{Ethics}\label{ethics}

NA

Long Definition:

Deep Dive:

NA

Contributors: NA

Tags: NA

\bookmarksetup{startatroot}

\chapter{F}\label{f}

\section{Feedback}\label{feedback}

NA

Long Definition:

Deep Dive:

NA

Contributors: NA

Tags: NA

\bookmarksetup{startatroot}

\chapter{G}\label{g}

\section{Gamification}\label{gamification}

NA

Long Definition:

Deep Dive:

NA

Contributors: NA

Tags: NA

\section{Glasgow Specific Terms}\label{glasgow-specific-terms}

NA

Long Definition:

Deep Dive:

NA

Contributors: NA

Tags: NA

\bookmarksetup{startatroot}

\chapter{H}\label{h}

\section{Hidden Curriculum}\label{hidden-curriculum}

The unwritten knowledge of the workings of an institution generally
learned through experience

Long Definition:

Deep Dive:

NA

Contributors: Phil McAleer

Tags: NA

\bookmarksetup{startatroot}

\chapter{K}\label{k}

\section{Knowledge Exchange}\label{knowledge-exchange}

NA

Long Definition:

Deep Dive:

NA

Contributors: NA

Tags: NA

\bookmarksetup{startatroot}

\chapter{L}\label{l}

\section{Learning}\label{learning}

NA

Long Definition:

Deep Dive:

NA

Contributors: NA

Tags: NA

\section{Learning Analytics}\label{learning-analytics}

NA

Long Definition:

Deep Dive:

NA

Contributors: NA

Tags: NA

\section{Lti}\label{lti}

NA

Long Definition:

Deep Dive:

NA

Contributors: NA

Tags: NA

\section{Lts}\label{lts}

Learning, teaching and scholarship

Long Definition:

Deep Dive:

A career track at the University of Glasgow
https://www.gla.ac.uk/media/Media\_499574\_smxx.pdf

Contributors: Alison McCandlish

Tags: NA

\bookmarksetup{startatroot}

\chapter{M}\label{m}

\section{Mann-Whitney U Test}\label{mann-whitney-u-test}

NA

Long Definition:

Deep Dive:

NA

Contributors: NA

Tags: NA

\section{Mentimeter}\label{mentimeter}

An interactive computer programme tool which allows an audience to
interact with pre-defined (presenter designed) questions using their own
device.

Long Definition:

Deep dive: https://www.mentimeter.com/ is the main company website.
https://www.mentimeter.com/education shows education use cases.

The company describes the tool as a way to ``turn presentations into
conversations with interactive polls that engage meetings and
classrooms''

Deep Dive:

NA

Contributors: Alison McCandlish

Tags: NA

\section{Mixed Methods}\label{mixed-methods}

NA

Long Definition:

Deep Dive:

NA

Contributors: NA

Tags: NA

\section{Moodle}\label{moodle}

NA

Long Definition:

Deep Dive:

NA

Contributors: NA

Tags: NA

\section{Multidisciplinary}\label{multidisciplinary}

NA

Long Definition:

Deep Dive:

NA

Contributors: NA

Tags: NA

\bookmarksetup{startatroot}

\chapter{O}\label{o}

\section{Ontology}\label{ontology}

NA

Long Definition:

Deep Dive:

NA

Contributors: NA

Tags: NA

\section{Open Science}\label{open-science}

NA

Long Definition:

Deep Dive:

NA

Contributors: NA

Tags: NA

\section{Open Source}\label{open-source}

NA

Long Definition:

Deep Dive:

NA

Contributors: NA

Tags: NA

\section{Outputs}\label{outputs}

NA

Long Definition:

Deep Dive:

NA

Contributors: NA

Tags: NA

\bookmarksetup{startatroot}

\chapter{P}\label{p}

\section{P-Value}\label{p-value}

The probability of your data (or more extreme), assuming the null
hypothesis is true.

Long Definition:

Under frequentist inferential statistics, the p-value represents the
probability of your data (or more extreme), assuming that the null
hypothesis is true. Informally, you can see it as a measure of surprise,
where a small p-value means your data would be surprising under the
null. Conversely, a large p-value means your data would not be
surprising under the null. The idea behind this technique is helping you
make decisions where you can either reject the null or retain the null.
Given an alpha value (often .05 or 5\%), rejecting the null means you
conclude there is an effect, whereas retaining the null means you do not
conclude there is an effect.

Deep Dive:

NA

Contributors: James Bartlett; Phil McAleer

Tags: Statistics

\section{Passive Learning}\label{passive-learning}

In contrast to active learning, this term is associated with students
being exposed to a predominantly an information transmission approach
where they have no agency or investment in their learning.

Long Definition:

Students cramming for a test may adopt a superficial approach to
learning where the goal or motivation is to regurgitate with the purpose
of passing an exam.

Deep Dive:

NA

Contributors: NA

Tags: NA

\section{Pedagogy}\label{pedagogy}

NA

Long Definition:

Deep Dive:

NA

Contributors: NA

Tags: NA

\section{Peer Assessment}\label{peer-assessment}

NA

Long Definition:

Deep Dive:

NA

Contributors: NA

Tags: NA

\section{Personalised Learning}\label{personalised-learning}

NA

Long Definition:

Deep Dive:

NA

Contributors: NA

Tags: NA

\section{Problem-Based Learning}\label{problem-based-learning}

NA

Long Definition:

Deep Dive:

NA

Contributors: NA

Tags: NA

\bookmarksetup{startatroot}

\chapter{Q}\label{q}

\section{Qualitative}\label{qualitative}

NA

Long Definition:

Deep Dive:

NA

Contributors: NA

Tags: NA

\section{Quantitative}\label{quantitative}

NA

Long Definition:

Deep Dive:

NA

Contributors: NA

Tags: NA

\bookmarksetup{startatroot}

\chapter{R}\label{r}

\section{Red Brick University}\label{red-brick-university}

NA

Long Definition:

Deep Dive:

NA

Contributors: NA

Tags: NA

\section{Reflective Practice}\label{reflective-practice}

NA

Long Definition:

Deep Dive:

NA

Contributors: NA

Tags: NA

\section{Regression}\label{regression}

NA

Long Definition:

Deep Dive:

NA

Contributors: NA

Tags: NA

\section{Research-Led Teaching}\label{research-led-teaching}

NA

Long Definition:

Deep Dive:

NA

Contributors: NA

Tags: NA

\section{Russell Group}\label{russell-group}

NA

Long Definition:

Deep Dive:

NA

Contributors: NA

Tags: NA

\bookmarksetup{startatroot}

\chapter{S}\label{s}

\section{Sample Size}\label{sample-size}

NA

Long Definition:

Deep Dive:

NA

Contributors: NA

Tags: NA

\section{Scholarship}\label{scholarship}

NA

Long Definition:

Deep Dive:

NA

Contributors: NA

Tags: NA

\section{Sotl}\label{sotl}

An abbreviation for the phrase scholarship of learning and teaching

Long Definition:

The scholarship of learning and teaching is

Deep Dive:

NA

Contributors: NA

Tags: NA

\section{Student Voice}\label{student-voice}

NA

Long Definition:

Deep Dive:

NA

Contributors: NA

Tags: NA

\section{Sustainable}\label{sustainable}

NA

Long Definition:

Deep Dive:

NA

Contributors: NA

Tags: NA

\section{Synchronous}\label{synchronous}

NA

Long Definition:

Deep Dive:

NA

Contributors: NA

Tags: NA

\bookmarksetup{startatroot}

\chapter{T}\label{t}

\section{Teaching}\label{teaching}

NA

Long Definition:

Deep Dive:

NA

Contributors: NA

Tags: NA

\section{Team Teaching}\label{team-teaching}

NA

Long Definition:

Deep Dive:

NA

Contributors: NA

Tags: NA

\bookmarksetup{startatroot}

\chapter{V}\label{v}

\section{Virtual Reality}\label{virtual-reality}

NA

Long Definition:

Deep Dive:

NA

Contributors: NA

Tags: NA

\section{Vle}\label{vle}

NA

Long Definition:

Deep Dive:

NA

Contributors: NA

Tags: NA

\bookmarksetup{startatroot}

\chapter{W}\label{w}

\section{Work-Integrated Learning}\label{work-integrated-learning}

NA

Long Definition:

Deep Dive:

NA

Contributors: NA

Tags: NA

\bookmarksetup{startatroot}

\chapter*{References}\label{references}
\addcontentsline{toc}{chapter}{References}

\markboth{References}{References}

\phantomsection\label{refs}
\begin{CSLReferences}{0}{1}
\end{CSLReferences}




\end{document}
